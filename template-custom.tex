\documentclass{article}
\usepackage{hyperref}
\usepackage{amsmath}
\usepackage{amsthm}
\usepackage{amssymb}

\begin{document}

\title{Calcul numeric - tem\u{a} de laborator}

\author{Andrei - Gabriel GAVRIL\u{A}, grupa 10LD521}

\date{Februarie - Mai 2023}

\maketitle

\section*{Enun\c{t}: Capitolul 8, Subcapitolul I, Problema 13}

S\u{a} se calculeze factorizarea LU a matricei:
\begin{center}
$
A=\begin{pmatrix}
1 & 2 & 3 & 4\\
2 & 3 & 4 & 1\\
3 & 4 & 1 & 2\\
4 & 1 & 2 & 3
\end{pmatrix}
$
\end{center}

\section*{Solu\c{t}ie}

\begin{center}
    A = [1,2,3,4; 2,3,4,1; 3,4,1,2; 4,1,2,3];
\end{center}
\begin{center}
    [L,U,P]=lu(A)
\end{center}

\begin{center}
$
L =\begin{pmatrix}

    1.0000  &       0  &       0  &       0\\
    0.7500  &  1.0000  &       0  &       0\\
    0.5000  &  0.7692  &  1.0000  &       0\\
    0.2500  &  0.5385  &  0.8182  &  1.0000
\end{pmatrix}
$
\end{center}

\begin{center}
$
U =\begin{pmatrix}
  4.0000  &  1.0000  &  2.0000  &  3.0000\\
    0  &  3.2500  & -0.5000 &  -0.2500\\
    0  & 0  &  3.3846  & -0.3077\\
    0  & 0  &  0  &  3.6364
\end{pmatrix}
$
\end{center} 

\begin{center}
$
P =\begin{pmatrix}
  0  &   0  &   0   & 1\\
  0  &   0  &   1   &  0\\
  0  &   1  &   0   &  0\\
  1  &   0  &   0   &  0
\end{pmatrix}
$
\end{center}
    

\section*{Observa\c{t}ii}

...

\end{document}
