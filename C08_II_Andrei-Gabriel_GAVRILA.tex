\documentclass{article}
\usepackage{hyperref}
\usepackage{amsmath}
\usepackage{amsthm}
\usepackage{amssymb}

\begin{document}

\title{Calcul numeric - tem\u{a} de laborator}

\author{Andrei - Gabriel GAVRIL\u{A}, grupa 10LD521}

\date{Februarie - Mai 2025}

\maketitle

\section*{Enun\c{t}: Capitolul 8, Subcapitolul II, Problema 1}

S\u{a} se calculeze descompunerea / factorizarea QR a matricei:
\begin{center}
$
A=\begin{pmatrix}
  4 & 5 & 2 \\
  3 & 0 & 3 \\
  0 & 4 & 6
\end{pmatrix}
$
\end{center}

\section*{Solu\c{t}ie}

\begin{center}
A = [4, 5, 2; 3, 0, 3; 0, 4, 6]
\end{center}
\begin{center}
[Q, R] = qr(A)
\end{center}

\begin{center}
$
Q =\begin{pmatrix}
  -0.8000 &    0.3600 &   -0.4800 \\
  -0.6000 &   -0.4800 &    0.6400 \\
        0 &    0.8000 &    0.6000 \\
\end{pmatrix}
$
\end{center}

\begin{center}
$
R =\begin{pmatrix}
  -5.0000 &   -4.0000 &   -3.4000 \\
        0 &    5.0000 &    4.0800 \\
        0 &         0 &    4.5600
\end{pmatrix}
$
\end{center} 
    
\section*{Observa\c{t}ii}

...

\end{document}
